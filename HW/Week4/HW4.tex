\documentclass{article} % Uses 10pt

\usepackage{fancyhdr}
\usepackage{helvet}
\usepackage{url}
\usepackage{amsmath,amssymb,kbordermatrix,graphicx}
\usepackage{tikz}
\usetikzlibrary{arrows,positioning, %
                petri,%
                topaths,%
                shapes,%
                snakes,%
                automata,%
                backgrounds}%
\textwidth6.5in
\textheight=9.7in  % text height can be bigger for a longer letter
\setlength{\topmargin}{-0.3in}
\addtolength{\topmargin}{-\headheight}
\addtolength{\topmargin}{-\headsep}
\headsep = 20pt

\DeclareMathOperator*{\argmax}{arg\,max}
\DeclareMathOperator*{\argmin}{arg\,min}
\DeclareMathOperator{\E}{\mathbb{E}}


\renewcommand{\baselinestretch}{1.4}

\setlength{\oddsidemargin}{0in}

\oddsidemargin  0.0in \evensidemargin 0.0in

%%\parindent0em

\pagestyle{fancy}
\renewcommand{\headrulewidth}{0.0pt}
\lhead{}
\chead{}
\rhead{}
\lfoot{}
\rfoot{}
\cfoot{}

\newcommand{\ind}{\mbox{$\perp \kern-5.5pt \perp$}}

\newcommand{\sectionname}[1]{\vspace{0.5cm} \noindent {\bf #1}}

\begin{document}

\begin{center}
  \textbf{\large Stat 516, Homework 4}
\end{center}
\sectionname{Due date}:  Tuesday, October 31.
\hfill {\bf Note}: Do \emph{individual} work.

\begin{enumerate}

\item
  \begin{enumerate}
  \item Prove that recurrence is a communication
    class property: $i \leftrightarrow j$ and $i$ is recurrent
    $\Rightarrow$ $j$ is recurrent.

\begin{flalign*}
  P_i(X_k=j) & > 0 \\
  P_j(X_m=i) & > 0 \\
  P_{ij}^{(k)} > 0, & ~ P_{ii}^{(n)} > 0, ~ P_{ji}^{(m)} > 0 \\
  \sum_{n=1}^{\infty} P_{ii}^{(n)} & = \infty = \sum_{n=1}^{\infty} P_i(X_n=i) \\
  P_{jj}^{(k+m+n)} & = P_{ij}^{(k)} P_{ii}^{(n)} P_{ji}^{(m)} \\
  & = P_{ij}^{(k)} P_{ji}^{(m)} \sum_{n=1}^{\infty} P_{ii}^{(n)} = \infty
\end{flalign*}

  \item Is the Markov chain with the transition graph below
    irreducible?  Explain. Find its period and its cyclic classes.
    (Note: the graph has no self-loops, so the transitions from a
    state $i$ to itself have probability zero.)

The markov chain is irreducible because each state can be reached by every other
state given enough time. The cyclic classes of the markov chain are
$\{3,5,6\} ~ \{1,2\} ~ \{4,7\}$ which makes $d$ and the periodicity of the
graph $3$.

  \end{enumerate}




\item Consider a fair simple random walk on $\mathbb{Z}^2$, that is,
  each state $(i,j)$ has four neighbors (in-/decrease $i$ or $j$ by 1)
  and the four transitions to these neighbors are equally probable.
  Show that the state (0,0) is recurrent.  Is it positive recurrent?

Let $\rho_{k}^{(n)}$ be the probability of returning to the k-dimensional
origin after n steps.

Given the 1D random walk

$$
\rho_{1}^{(n)} = {2n \choose n} 2^{-2n} \sim \frac{1}{\sqrt{\pi n}}
$$

then for the 2D random walk

\begin{flalign*}
  \rho_{2}^{(n)} & = 4^{-2n} \sum_{m=0}^{n} \frac{2n!}{m!m!(n-m)!(n-m)!} \\
  & = 4^{-2n} {2n \choose n} \sum_{m=0}^{n} {n \choose m} {n \choose n-m} \\
  & = \Big( 2^{-2n} {2n \choose n} \Big)^2 \\
  & = \big( \rho_1^{(n)} \big)^2 \\
  & \sim \frac{1}{\pi n}
\end{flalign*}

which means the series converges as $\rho_{2}^{(n)}$ is less than 1 as n
increases.

\item %(Br\'emaud 3.2.4)
  For $\alpha,\beta,\gamma\in(0,1)$, consider the transition probability matrix
\[
\mathbf{P}=\begin{pmatrix}
1-\alpha & \alpha & 0\\
0 & 1-\beta & \beta\\
\gamma & 0 & 1-\gamma
\end{pmatrix},
\]
   Show that the associated Markov
chain is irreducible and compute its stationary distribution.  Do it
first by solving the linear equation system from the definition of a
stationary distribution.  Then derive the stationary distribution again
using the ``regenerative form of invariant measures.''

By linear equations...

\begin{flalign*}
  \pi_1 & = \frac{\gamma \pi_3}{\alpha} \\
  \pi_2 & = \frac{\alpha \pi_1}{\beta} \\
  \pi_3 & = \frac{\beta \pi_2}{\alpha} \\
  1 & = \pi_1 + \pi_2 + \pi_3 \\
  1 & = \frac{\gamma \pi_3}{\alpha} + \frac{\gamma \pi_3}{\beta} + \pi_3 \\
  1 & = \pi_3 (\frac{\gamma}{\alpha} + \frac{\gamma}{\beta} + 1) \\
  \pi_3 & = (\frac{\gamma}{\alpha} + \frac{\gamma}{\beta} + 1)^{-1} \\
  \pi_1 & = (\frac{\alpha}{\gamma} + \frac{\alpha}{\beta} + 1)^{-1} \\
  \pi_2 & = (\frac{\beta}{\alpha} + \frac{\beta}{\gamma} + 1)^{-1} \\
\end{flalign*}

By regenerative form of interval measures...

\begin{flalign*}
  E[X] & = P(X \geq 1) + P(X \geq 2) + P(X \geq 3) + ... \\
  E_1[X_1] & = 1 + 0 + 0 + ... \\
  E_1[X_1] & = 1 \\
  E_1[X_2] & = \alpha + \alpha (1 - \beta) + \alpha (1 - \beta)^2 + ... \\
  E_1[X_2] & = \frac{\alpha}{\beta} \\
  E_1[X_3] & = \alpha + \alpha (1 - \gamma) + \alpha (1 - \gamma)^2 + ... \\
  E_1[X_3] & = \frac{\alpha}{\gamma} \\
  \pi_1 & = E_1[X_1] * \pi_1 = (\frac{\alpha}{\gamma} + \frac{\alpha}{\beta} + 1)^{-1} \\
  \pi_2 & = E_1[X_2] * \pi_1 = (\frac{\beta}{\alpha} + \frac{\beta}{\gamma} + 1)^{-1} \\
  \pi_3 & = E_1[X_3] * \pi_1 = (\frac{\gamma}{\alpha} + \frac{\gamma}{\beta} + 1)^{-1} \\
\end{flalign*}

\item (Guttorp 2.2.14) Consider a Markov chain on a finite state space
  $S$, with a t.p.m.\ that has entries $p_{ij}>0$ for all $i,j\in S$.
  Let $\pi$ be the stationary distribution.  Show that the $n$-step
  transition probabilities $p^{(n)}_{ij}$ satisfy
  \[
  \left| p^{(n)}_{ij} - \pi_j \right| \;\le\; (1-d\delta)^n,
  \]
  where $d=|S|$ and $\delta=\min p_{ij}$.\\[0.05cm]
  {\em Hint:} Divide the terms in the equation $\sum_j
  (p_{ij}-p_{kj})=0$ into those with positive and those with negative
  values.  The sum of the positive terms is bounded by $1-d\delta$.
  Now bound $\max_i p^{(n+1)}_{ij} -\min_i p^{(n+1)}_{ij}$ using the
  Chapman-Kolmogorov equations (i.e., one-step analysis).

\begin{flalign*}
  \pi_j & = \sum_{k \in S} \pi_k p_{kj} \\
  p^{(n)}_{ij} - \pi_j & = - \sum_{k \in S} \pi_k \Big(
  p^{(n)}_{kj} - p^{(n)}_{ki} \Big)
\end{flalign*}

\end{enumerate}

\end{document}

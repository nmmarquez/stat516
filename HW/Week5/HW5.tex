\documentclass{article} % Uses 10pt

\usepackage{fancyhdr}
\usepackage{helvet}
\usepackage{url}
\usepackage{amsmath,amssymb,kbordermatrix,graphicx}
\textwidth6.5in
\textheight=9.7in  
\setlength{\topmargin}{-0.3in} 
\addtolength{\topmargin}{-\headheight}
\addtolength{\topmargin}{-\headsep}
\headsep = 20pt

\DeclareMathOperator*{\argmax}{arg\,max}
\DeclareMathOperator*{\argmin}{arg\,min}
\DeclareMathOperator{\E}{\mathbb{E}}


\renewcommand{\baselinestretch}{1.4} 

\setlength{\oddsidemargin}{0in}

\oddsidemargin  0.0in \evensidemargin 0.0in 

%%\parindent0em

\pagestyle{fancy}
\renewcommand{\headrulewidth}{0.0pt}
\lhead{}
\chead{}
\rhead{} 
\lfoot{} 
\rfoot{} 
\cfoot{}

\newcommand{\ind}{\mbox{$\perp \kern-5.5pt \perp$}}

\newcommand{\sectionname}[1]{\vspace{0.5cm} \noindent {\bf #1}}

\begin{document}

\begin{center}
  \textbf{\large Stat 516, Homework 5}
\end{center}
\sectionname{Due date}:  Thursday, November 16.

\noindent
\noindent
{\bf Note}: Do this homework in \emph{pairs}; two students turning in
a single joint solution.  No two Statistics, Biostatistics, QERM, EE,
Econ, AMath, \dots students may work together.  (Exceptions as needed based on the make-up of the class.)

\begin{enumerate}
\item Consider the model $Y | \theta \sim \mbox{Poisson}(E \times \theta)$ where $E$ is a known ``expected number'' of cases, $Y$ is the count of disease cases and $\theta >0$ is the relative risk with $\theta=1$ corresponding to ``null'' risk.
\begin{enumerate}
\item Find expressions for the likelihood function $L(\theta)$, the log
  likelihood function $l(\theta)$, the score function $S(\theta)$ and
  Fisher's (expected) information $I(\theta)$.  Find the MLE and its variance.
\item Suppose we assume a prior of $\theta \sim \mbox{Gamma}(a,b)$ so that
$$p(\theta) = \frac{b^a}{\Gamma(a)}\theta^{a-1} \exp( - b \theta),$$
with $a,b>0$. Show that the posterior $\theta | y$ is also gamma and
find its parameters.
\item Close to a nuclear reprocessing plant 
  $y=4$ cases of
  leukemia were observed with an expected number of $E=0.25$.
  Give the MLE, its variance and a 95\%
  confidence interval based on a normal approximation.  
\item Find the $a$ and $b$ which give a gamma prior that assigns 0.9
  probability to the interval [0.1,10].
Find the posterior corresponding to this prior and generate samples from the posterior. Give a histogram of the posterior and state a 95\% credible interval.
\item Is there evidence of excess risk for these data? Discuss the differences between the likelihood-based and Bayesian analyses.
\end{enumerate}
\item Consider the Snoqualmie falls data available on the class
  website. 
  We will analyze the
  data for June across the 36 years; compare how the data was
  processed in examples in the lecture notes.
\begin{enumerate}
\item Using the multiple years of data estimate the transition
  probability matrix for the first-order Markov chain model, with
  $p_{12}$
  the probability of wet given dry and $p_{21}$
  the probability of dry given wet.  Obtain MLEs and 95\% asymptotic
  confidence intervals for $p_{12}$ and $p_{21}$.
\item Under the assumption of independent uniform priors, give the
  form of the posterior distributions for the parameters
  $p_{12},p_{21}$. Obtain the posterior and report posterior medians
  and 95\% credible intervals for each of the two parameters.
\item Test the null of independence versus the Markov model, using a likelihood ratio test and a Bayes factor.
\item Now consider the null of common transition probabilities across
  years versus the alternative that each year possesses its own pair
  of probabilities. First, estimate the probabilities in each year and
  provide a figure of the logits of the probabilities,
  i.e.~$\log\left(\frac{p_{12}}{1-p_{12}}\right)$
  and $\log\left(\frac{p_{21}}{1-p_{21}}\right)$,
  plotted versus year. Second, carry out a likelihood ratio test to
  formally test the hypothesis. Third, evaluate the Bayes factor to
  examine the evidence for each of the two hypotheses. What do you
  conclude?
\end{enumerate}
\item For ternary observations (i.e., $s=3$ states), consider a Markov
  chain model in which the $3\times 3$ transition probability matrix
  $\mathbf{P}=(p_{ij})$ is unrestricted.  With the row sums being one,
  we may think of there being 6 parameters, say $p_{ij}$ with $1\le
  i\le 3$ and $1\le j\le 2$.
  \begin{enumerate}
  \item Specialize the result on likelihood inference for parametric
    transition probabilities to this case and write out the $6\times
    6$ joint asymptotic covariance matrix for the MLE
    \[
    \left(\hat p_{11},\hat p_{12},\, \hat p_{21},\hat
      p_{22}, \,\hat p_{31},\hat p_{32}\right),
    \]
    giving formulas in terms of $\mathbf{P}$ and the stationary
    distribution $\pi$ (which is assumed to exist).
  \item Write out the $9\times 9$ joint asymptotic covariance matrix
    for the MLE
    \[
    \left(\hat p_{11},\hat p_{12},\hat p_{13},\, \hat p_{21},\hat
      p_{22},\hat p_{23},\, \hat p_{31},\hat p_{32},\hat p_{33}\right).
    \]
  \item Discuss the asymptotic dependences you find.
  \end{enumerate}
\end{enumerate}

\end{document}

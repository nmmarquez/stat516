\documentclass{article} % Uses 10pt

\usepackage{fancyhdr}
\usepackage{helvet}
\usepackage{url}
\usepackage{amsmath,amssymb,kbordermatrix,graphicx}
\textwidth6.5in
\textheight=9.7in  % text height can be bigger for a longer letter
\setlength{\topmargin}{-0.3in}
\addtolength{\topmargin}{-\headheight}
\addtolength{\topmargin}{-\headsep}
\headsep = 20pt

\DeclareMathOperator*{\argmax}{arg\,max}
\DeclareMathOperator*{\argmin}{arg\,min}


\renewcommand{\baselinestretch}{1.4}

\setlength{\oddsidemargin}{0in}

\oddsidemargin  0.0in \evensidemargin 0.0in

%%\parindent0em

\pagestyle{fancy}
\renewcommand{\headrulewidth}{0.0pt}
\lhead{}
\chead{}
\rhead{}
\lfoot{}
\rfoot{}
\cfoot{}

\newcommand{\ind}{\mbox{$\perp \kern-5.5pt \perp$}}

\newcommand{\sectionname}[1]{\vspace{0.5cm} \noindent {\bf #1}}

\begin{document}

\begin{center}
  \textbf{\large Stat 516, Homework 6}
\end{center}
\sectionname{Due date}:  Tuesday, November 28.

\noindent
{\bf Note}:  Do this homework \emph{individually}.  Do not include any
R code in your main handout -- just include as an appendix, in compact
form.

\begin{enumerate}
\item
  \begin{enumerate}
  \item As an illustration of rejection sampling, show how one can
    generate a draw from $N(0,1)$ by using a Cauchy proposal
    distribution.  What is the acceptance rate of your sampler?  What
    is the mean number of trials until acceptance of the Cauchy draw?
    (The Cauchy has density $1/\left[\pi(1+x^2)\right]$.)

\begin{flalign*}
  f(x) & = \frac{1}{\sqrt{2 \pi}} \text{exp} \Big( - \frac{x^2}{2} \Big) \\
  g(x) & = \frac{1}{\pi (1+x^2)} \\
  \frac{f}{g}(x) & = \frac{\pi (1+x^2) \text{exp} \big( - \frac{x^2}{2} \big)}
                    {\sqrt{2 \pi}} \\
  \frac{d}{dx} & = \frac{\pi}{\sqrt{2 \pi}}
                   - \text{exp} \Big( - \frac{x^2}{2}\Big) x (x^2 - 1) \\
  \text{set } \frac{d}{dx} \text{ to } 0 \text{ to find } x_M \text{...} \\
  x_M & = \pm 1 \\
  M & = \frac{2 \pi}{\sqrt{2 \pi}} \exp \Big( \frac{-1}{2} \Big) \approx 1.52 \\
  p_a & = \frac{\int_{-\infty}^{\infty} f(x)}{M} = M^{-1} \approx .66 \\
  \text{Expected number of trails is then} \\
  E[X] & = p(X \geq 1) + p(X \geq 2) + p(X \geq 3) + \dots \\
  & = 1 + (1 - p_a) + (1 - p_a)^2 + \dots \\
  & = \frac{1}{p_a} = M
\end{flalign*}

  \item Again you wish to generate a draw from $N(0,1)$.  But instead
    of using the standard Cauchy proposal from (a) you are using a
    scaled Cauchy with density
    $1/\left[\pi\gamma(1+(x/\gamma)^2)\right]$, where $\gamma>0$ is a
    scale parameter.  How would optimally choose $\gamma$?
  \item Is it possible to generate a draw from a Cauchy distribution
    using $N(0,1)$ as proposal distribution?
  \end{enumerate}

\item
 In this question we will analyze data on 10 power plant pumps using a
 Poisson gamma model. The number of failures $Y_i$ is assumed to
 follow a Poisson distribution
\[ Y_i | \theta_i \sim_{ind} \mbox{Poisson}(\theta_i t_i),\quad i=1,...,10\]
where $\theta_i$ is the failure rate for pump $i$ and $t_i$ is the length of operation time of the pump (in 1000s of hours). The data is shown below.


\begin{table}[h]
\begin{center}
\begin{tabular}{c|cccccccccc}
Pump & 1&2&3&4&5&6&7&8&9&10\\ \hline
$t_i$& 94.3&15.7&62.9&126&5.24&31.4&1.05& 1.05& 2.1&10.5\\ \hline
$y_i$&5&1&5&14&3&19&1&1&4&22
\end{tabular}
\end{center}
\end{table}

A conjugate gamma prior distribution is adopted for the failure rates:
\[ \theta_i |\alpha,\beta \sim_{iid} \mbox{Gamma}(\alpha,\beta),\quad i=1,...,10,
\]
with a hyperprior under which $\alpha$ and $\beta$ are independent and
\begin{align*}
\alpha &\sim \mbox{Exponential}(1), &
\beta &\sim \mbox{Gamma}(0.1,1).
\end{align*}
\begin{enumerate}
\item Carefully show, using Bayes theorem and the conditional independencies in the model description, that the posterior distribution is given by:
\begin{equation}\label{eq:postpump}
p( \alpha,\beta,\boldsymbol{\theta} | \mathbf{y} ) \propto \prod_{i=1} \left\{ \Pr( y_i | \theta_i) \times
p(\theta_i| \alpha,\beta ) \right\} \times \pi(\alpha,\beta)
\end{equation}
where $\boldsymbol{\theta}=(\theta_1,...,\theta_{10})$.
\item By using (\ref{eq:postpump}) write out the steps of a Metropolis-Hastings within Gibbs
  sampling algorithm to analyze these data.

Hint: First write down the forms for $(\alpha|\beta,\boldsymbol{\theta},\mathbf{y})$, $(\beta|\alpha,\boldsymbol{\theta},\mathbf{y})$, $(\theta_i | \boldsymbol{\theta}_{-i},\alpha,\beta,\mathbf{y})$, for $i=1,...,10$.
\item Apply your algorithm to the pump data and give histogram
  representations of the univariate posterior distributions for
  $\alpha$ and $\beta$ and a scatterplot representing the bivariate
  posterior distribution.
\item Analytically integrate $\theta_i$, $i=1,...,10$ from the
  posterior (\ref{eq:postpump}) and hence give the form, up to
  proportionality, of the posterior $p(\alpha,\beta|\mathbf{y})$.
\item Construct a Metropolis-Hastings algorithm, to provide a Markov
  chain with stationary distribution the posterior
  $p(\alpha,\beta|\mathbf{y})$.
\item Implement the algorithm of the previous part, and provide the
  same univariate and bivariate posteriors that were produced in part (c).
\item How can you obtain samples from $p(\theta_i|\mathbf{y})$, based on samples from $p(\alpha,\beta|\mathbf{y})$?
\end{enumerate}



\end{enumerate}

\end{document}

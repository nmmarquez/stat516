\mathbf{y}\documentclass{article} % Uses 10pt
%\input{definitions}

\usepackage{fancyhdr}
\usepackage{helvet}
\usepackage{url}
\usepackage{amsmath,amssymb,kbordermatrix,graphicx}
\textwidth6.5in
\textheight=9.7in  % text height can be bigger for a longer letter
\setlength{\topmargin}{-0.3in}
\addtolength{\topmargin}{-\headheight}
\addtolength{\topmargin}{-\headsep}
\headsep = 20pt

\DeclareMathOperator*{\argmax}{arg\,max}
\DeclareMathOperator*{\argmin}{arg\,min}


\renewcommand{\baselinestretch}{1.4}

\setlength{\oddsidemargin}{0in}

\oddsidemargin  0.0in \evensidemargin 0.0in

%%\parindent0em

\pagestyle{fancy}
\renewcommand{\headrulewidth}{0.0pt}
\lhead{}
\chead{}
\rhead{}
\lfoot{}
\rfoot{}
\cfoot{}

\newcommand{\ind}{\mbox{$\perp \kern-5.5pt \perp$}}

\newcommand{\sectionname}[1]{\vspace{0.5cm} \noindent {\bf #1}}

\begin{document}

\begin{center}
  \textbf{\large Stat 516, Homework 7}
\end{center}
\sectionname{Due date}:  Thursday, December 7.

\noindent
{\bf Note}: Do this homework \emph{in pairs} under the usual rules.


\begin{enumerate}



\item This problem is a simple illustration of how the
  introduction of suitable complete data $\mathbf{x}$ can help address
  questions about observed data $\mathbf{y}$.
\begin{enumerate}
\item
Consider data $\mathbf{y}=[y_1,y_2,y_3,y_4]=[125,18,20,34]$ which are multinomially distributed and for which a genetic model suggests the cell probabilities are
$$\left[ \frac{1}{2} + \frac{\pi}{4},\frac{1-\pi}{4},\frac{1-\pi}{4},\frac{\pi}{4}
\right]
$$
with $0 \leq \pi \leq 1$.
Hence,
$$p(\mathbf{y}|\pi) = \frac{(y_1+y_2+y_3+y_4)!}{y_1!y_2!y_3!y_4!}
\left(\frac{1}{2}+ \frac{\pi}{4}\right)^{y_1}\left(\frac{1-\pi}{4}\right)^{y_2}\left(\frac{1-\pi}{4}\right)^{y_3}\left(\frac{\pi}{4}\right)^{y_4}.$$

Develop an EM algorithm for computation of the maximum likelihood
estimate of $\pi$.  To this end, consider {\it complete data} $x_1$,
$x_2$, $x_3$, $x_4$, $x_5$ that follow a multinomial distribution with
cell probabilities
$$\left[ \frac{1}{2},\frac{\pi}{4},\frac{1-\pi}{4},\frac{1-\pi}{4},\frac{\pi}{4}
\right].
$$
Explain how you relate $x$ to $y$ and carefully write down the E
and M steps of the EM algorithm.  Implement the algorithm for the data
given above.
\smallskip

Considering four arbitrary positive counts, can the
likelihood function of the above model for $\mathbf{y}$ have more than one local
maximum?

\item Now consider a Bayesian analysis with a Be$(1,1)$ prior on
  $\pi$. Write out the details of an auxiliary variable scheme that
  samples $p(\mathbf{x}|\mathbf{y},\pi)$ and $p(\pi|\mathbf{x})$. Implement this algorithm
  for the given data.
\end{enumerate}

\item In this problem you are asked to experiment with the
  forward-backward algorithm for HMMs, which computes the conditional
  probabilities $p(z_i\,|\,x)$ for each one of the hidden states
  $z_i$.  To this end, consider the `Occasionally dishonest Casino'
  example. Assume that 1=fair dice, 2=loaded dice and use transition
  probabilities
  \[
  \mathbf{P} =
  \kbordermatrix{ &1&2\\
    1&0.95 & 0.05 \\
    2&0.10 & 0.90
  },
  \]
  emission probabilities
  \[
  \mathbf{E} =
  \kbordermatrix{ & 1&2&3&4&5&6\\
    1 &\frac{1}{6} & \frac{1}{6} &  \frac{1}{6} & \frac{1}{6} &
    \frac{1}{6} &  \frac{1}{6}  \\
    2&\frac{1}{10} & \frac{1}{10} & \frac{1}{10} &
    \frac{1}{10} & \frac{1}{10}& \frac{1}{2}
  },
  \]
  and initial distribution $\boldsymbol{\nu} = (\frac{1}{2}, \frac{1}{2})$.
  \begin{enumerate}
  \item Code up your own implementation of the forward-backward
    algorithm as first derived in class, that is, simply compute
    probabilities without any rescaling or considering logarithms.
    Run the algorithm for different simulated data sets and report on
    the sequence lengths for which you start experiencing numerical
    problems.  (In R, the command {\tt sample} is useful for such
    simulations.  Also be sure to set a seed for random number
    generation via the {\tt set.seed} command in order to be able to
    reproduce your simulation results.)
  \item Modify your implementation from (a) to compute all
    probabilities on a log-scale using the fact
    \[
    \log(a+b) = \log(a) + \log\big(1+e^{\log(b)-\log(a)}\big).
    \]
    Does this resolve possible numerical issues with longer sequences
    when choosing $a$ to be the larger of any two summands?
  \item Modify your implementation from (a) to work with rescaled
    probabilities as discussed in class/in the handout from the book
    by Bishop.  Does this resolve possible numerical issues with
    longer sequences?  Compare to your results in (b).
  \end{enumerate}


\item (Guttorp Exercise 2.14.D8) The data in Guttorp's data set 3
  consists of four years of hourly average wind directions a the
  Koeberg weather station in South Africa.\footnote{The data set is
    available at {\tt
      http://www.stat.washington.edu/peter/book.data/set3}
  } The period covered is 1 May 1985 through 30
  April 1989.  The average is a vector average, categorized into 16
  wind directions from N (1), NNE (2), \dots to NNW (16).  Analyze the
  data using a hidden Markov model, with 2 hidden states, and
  conditionally upon the state a multinomial observation with 16
  categories.  Does the fitted model separate out different weather
  patterns?  Also estimate the underlying states, and look for
  seasonal behavior in the sequence of states.

  I strongly encourage you to write your own implementations of the
  relevant algorithms but you are also allowed to use existing
  routines in R or similar software to do this problem; in that case
  indicate which software you used.

  Provide a concise, well-written report of your work, showing
  graphics to support your conclusions.  Any plots you include should
  have clearly labelled axes.  You do not need to turn in your
  computer code as part of your solution but you should keep your code
  and be ready to email it to the grader/instructor if contacted about
  it after the homework is due.  (Zero points if you can't provide
  your own code when asked to email it or if the code does not give
  the results you described.)

\end{enumerate}

\end{document}
